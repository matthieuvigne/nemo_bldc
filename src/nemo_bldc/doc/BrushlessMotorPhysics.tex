\documentclass[a4paper,10pt]{article}
\usepackage[utf8]{inputenc}
\usepackage[T1]{fontenc} 
\usepackage{amssymb}
\usepackage{mathtools}
\usepackage{bm}
\usepackage{pdflscape}
\usepackage[british]{babel}
\usepackage{gensymb}
\usepackage{array}
\usepackage{float}
\usepackage{wrapfig}
\usepackage{hyperref}

\newtheorem{theorem}{Theorem}

\title{Physics of PMSM motors}
\author{Matthieu Vigne}
\date{September 2022}

\begin{document}

\maketitle
\tableofcontents

\section{Introduction}

The goal of this document is to summarize the equations of dynamics of a brushless motor. Specifically, we are here looking at
so-called PMSM (permanent magnet synchronous machine) - basically a brushless motor without cogging, i.e. with
sinusoidal back-EMF. 

These equations are derived from the following PhD thesis and books:

 - [1] Nicolas Henwood, Estimation en ligne de paramètres de machines
électriques pour véhicule en vue d’un suivi de la
température de ses composants
 
 - [2] L. Chédot, Contribution à l'étude des machines
synchrones à aimants permanents internes à
large espace de fonctionnement.
 Application à l'alterno-démarreur
 
 - [3] J. Chiasson. Modeling and High-Performance Control of Electric Machines. IEEE Press, 2005.

We use slightly different variables at some times, as will be outlined in what follows.

\section{Model equations}

\subsection{Fundamental equations}

We consider a star-shaped 3 phase brushless motor, with sinusoidal back-EMF. This motor is intrinsically characterized by only 4 fundamental constants:

\begin{itemize}
	\item $n$, the number of poles ; we denote $n_p \triangleq \frac{n}{2}$ the number of pole pairs.
	\item $R$, the per-phase resistance
	\item $L$, the per-phase inductance
	\item $\phi$, the rotor magnetic flux generated by the $n_p$ pole pairs - or equivalently, the back-EMF constant $k_e \triangleq \sqrt{\frac{2}{3}}$ (see [3, p.452]). Because we will be working with current-invariant Clarke-Park transforms, we will use $k_e$ over $\phi$.
\end{itemize}

Two other extrinsic limitations are present in a given setup:
\begin{itemize}
	\item $U_{bat}$, a maximum voltage, due to power supply's voltage
	\item $I_m$, a maximum RMS current: this current comes from thermal or power limitations.
\end{itemize}

We denote $\theta$ the mechanical angle of the motor, $\theta_e \triangleq \frac{\theta}{n_p}$ is the electrical angle. Let $\bm \omega \triangleq \dot{\theta}$.

Using Faraday's law of induction, the voltage across each phase writes:

\begin{equation}
	\left\{
	\begin{aligned}
		U_a &= R i_a + L \frac{d}{dt} i_a - k_e \omega \sin \theta_e \\
		U_b &= R i_b + L \frac{d}{dt} i_b -k_e \omega \sin (\theta_e - \frac{2 \pi}{3}) \\
		U_c &= R i_c + L \frac{d}{dt} i_c - k_e \omega \sin (\theta_e + \frac{2 \pi}{3})
	\end{aligned}
	\right.
	\label{ePhase}
\end{equation}

Notice that $k_e$ already takes into account the number of poles (i.e. there is no $n_p$ term in front). The $\sqrt{3/2}$ factor is explained in [8, p.452]

Note furthermore that this equation (with sinusoidal back-EMF) implies that the motor has no saliency, i.e. no cogging: this simplifies a bit the equations found in these theses, as $L_d = L_q$.

\subsection{From fixed to rotation frame: the Clark-Park}

The second Kirchhoff law states that $i_a = i_b = i_c$: thus, a PMSM is described by only two equations. Furthermore, a steady-state formulation can be obtained when working in the rotating frame, attached to the rotor: this is expressed by the Clarke and Park transform.

Note that, unlike Nicolas Henwood, we use \textbf{current-invariant} Clarke-Park transform, and not \textbf{power-invariant}. This is because motor controllers like Ingenia or Elmo use this current-invariant form ; this however introduces a scaling effect in the formulas.

The direct, current-invariant transform, thus defines a quadrature and direct current as:

\begin{equation}
	\begin{pmatrix}
		i_q \\
		i_q
	\end{pmatrix} = \frac{2}{3} \begin{pmatrix}
		\cos \theta_e & \sin \theta_e \\
		-\sin \theta_e & \cos \theta_e
	\end{pmatrix} \begin{pmatrix}
		1 & \frac{-1}{2} & - \frac{1}{2} \\
		0 & \frac{\sqrt{3}}{2}& - \frac{\sqrt{3}}{2}
	\end{pmatrix}
	\begin{pmatrix}
		i_a \\
		i_b \\
		i_c
	\end{pmatrix}
	\label{eCarkePark}
\end{equation}

One can easily show that, when $i_d = 0$, in sinusoidal regime the amplitude of $i_q$ and $i_{a,b,c}$ is the same, hence the name of current-invariant transform. Conversely, for power computation we have:

\begin{equation}
	i_a^2 + i_b^2 + i_c^2 = \frac{3}{2} (i_q^2 + i_d^2)
	\label{ePower}
\end{equation}

The inverse transform conversely writes:

\begin{equation}
	\begin{pmatrix}
		i_a \\
		i_b \\
		i_c
	\end{pmatrix} = 
	\frac{3}{2} \begin{pmatrix}
		\frac{2}{3} & 0 \\
		-\frac{1}{3} & \frac{\sqrt{3}}{3} \\
		-\frac{1}{3} & -\frac{\sqrt{3}}{3} \\
	\end{pmatrix}
	\begin{pmatrix}
		\cos \theta & -\sin \theta \\
		\sin \theta & \cos \theta
	\end{pmatrix}
	\begin{pmatrix}
		i_d \\
		i_q
	\end{pmatrix} 
	\label{eClarkeParkInverse}
\end{equation}

The same transform can be applied to the voltages as well, to define a direct and quadrature voltage:

\begin{equation}
	\begin{pmatrix}
		u_q \\
		u_q
	\end{pmatrix} = \frac{2}{3} \begin{pmatrix}
		\cos \theta_e & \sin \theta_e \\
		-\sin \theta_e & \cos \theta_e
	\end{pmatrix} \begin{pmatrix}
		1 & \frac{-1}{2} & - \frac{1}{2} \\
		0 & \frac{\sqrt{3}}{2}& - \frac{\sqrt{3}}{2}
	\end{pmatrix}
	\begin{pmatrix}
		U_a \\
		U_b \\
		U_c
	\end{pmatrix}
	\label{eCarkeParkV}
\end{equation}

\subsection{Torque Equation}

To compute the torque, [8, p.454] performs a direct computation of the Lorenz force applied by the rotor onto the coils, to obtain the following relationship:

\begin{equation}
	\tau = \sqrt{\frac{3}{2}} k_e i_q
\end{equation}


\subsection{Obtaining the electric differential equation}

To obtain a minimal representation of the system's dynamics, the idea is to use the Clarke-Park transform on the phase dynamics \eqref{ePhase}.

The derivative of \eqref{eCarkePark} yields

\begin{equation}
	\frac{d}{dt}
	\begin{pmatrix}
		i_q \\
		i_q
	\end{pmatrix} = n_p \omega \begin{pmatrix}
	i_q \\
	-i_d
\end{pmatrix} + \frac{2}{3} \begin{pmatrix}
		\cos \theta_e & \sin \theta_e \\
		-\sin \theta_e & \cos \theta_e
	\end{pmatrix} \begin{pmatrix}
		1 & \frac{-1}{2} & - \frac{1}{2} \\
		0 & \frac{\sqrt{3}}{2}& - \frac{\sqrt{3}}{2}
	\end{pmatrix}
	\frac{d}{dt}
	\begin{pmatrix}
		i_a \\
		i_b \\
		i_c
	\end{pmatrix}
	\label{eCarkeParkDiff}
\end{equation}

Then multiplying by $L$ using \eqref{ePhase} we get
\begin{equation}
	\begin{aligned}
	L \frac{d}{dt}
	\begin{pmatrix}
		i_q \\
		i_q
	\end{pmatrix} &= n_p L \omega \begin{pmatrix} i_q \\ -i_d \end{pmatrix} 
		+ \frac{2}{3} \begin{pmatrix}
		\cos \theta_e & \sin \theta_e \\
		-\sin \theta_e & \cos \theta_e
	\end{pmatrix} \begin{pmatrix}
		1 & \frac{-1}{2} & - \frac{1}{2} \\
		0 & \frac{\sqrt{3}}{2}& - \frac{\sqrt{3}}{2}
	\end{pmatrix}
	\begin{pmatrix}
		U_a - R i_a + k_e \omega \sin \theta_e \\
		U_b - R i_b + k_e \omega \sin (\theta_e - \frac{2 \pi}{3})\\
		U_c - R i_c + k_e \omega \sin (\theta_e + \frac{2 \pi}{3})
	\end{pmatrix}\\
	&= n_p L \omega \begin{pmatrix} i_q \\ -i_d \end{pmatrix} 
	 + \begin{pmatrix} u_q \\ u_d \end{pmatrix}
	 - R \begin{pmatrix} i_q \\ i_d \end{pmatrix}
	 +  k_e \omega \begin{pmatrix} 0 \\ -1 \end{pmatrix}
	\end{aligned}
\end{equation}

\subsection{Electrical bounds}

We now need to take into account the fact that current and voltage are both limited.

For the current, the relationship is simple: from \eqref{ePower} we get the following inequality

\begin{equation}
	i_q^2 + i_d^2 \leq 2 I_m
\end{equation}

Concerning the voltage, the limit arises from the fact that, at any given time, the phase-to-phase voltage
must be less than the input battery voltage $U$. This writes:
\begin{equation}
	\begin{aligned}
	|U_a - U_b| &\leq U_{bat} \\
	|U_a - U_c| &\leq U_{bat} \\
	|U_b - U_c| &\leq U_{bat} \\
	\end{aligned}
\end{equation}

Using the inverse Park-Clarke transform \eqref{eClarkeParkInverse}, this yields:

\begin{equation}
	\begin{aligned}
		\sqrt{3} |\cos \theta_e u_q + \sin \theta_e u_d| &\leq U_{bat} \\
		\sqrt{3} |\cos (\theta_e + \frac{2 \pi}{3}) u_q + \sin (\theta_e + \frac{2 \pi}{3}) u_d| &\leq U_{bat} \\
		\sqrt{3} |\cos (\theta_e - \frac{2 \pi}{3}) u_q + \sin (\theta_e - \frac{2 \pi}{3}) u_d| &\leq U_{bat} \\
	\end{aligned}
\end{equation}

We thus obtain inequalities that depend on $\theta$. A sufficient condition is thus for this inequality to work for every $\theta_e$. Computing the maximum of this function over $\theta_e$, we get the following inequality

\begin{equation}
	u_q^2 + u_d^2 \leq \frac{U_{bat}^2}{3}
\end{equation}

\subsection{Summary}

To sum up, a PMSM, seen as a torque source, is described by 

\begin{itemize}
	\item Two differential equations describing the evolution of the electrical states
	\item An algebraic equation giving the torque as a function of the current
	\item Two inequalities for the current and voltage limit.
\end{itemize}

This model thus writes

\begin{equation}
	\boxed{\begin{aligned}
		&\left\{
		\begin{aligned}
			L \frac{d i_d}{dt} &= - R i_d + n_p\omega L i_q + u_d \\
			L \frac{d i_q}{dt} &= - R i_q - \omega (n_p L i_d + k_e) + u_q\\
			\tau &= \frac{3}{2} k_e i_q
		\end{aligned}
		\right.\\
		\text{with:}
		&\left\{
		\begin{aligned}
			i_q^2 + i_d^2 &\leq 2 {I_m}^2 \\
			u_q^2 + u_d^2 &\leq \frac{U_{bat}^2}{3}
		\end{aligned}
		\right.
	\end{aligned}}
	\label{eFullModel}
\end{equation}


\section{Model analysis}

Having derived the equations of the PMSM, we now analyze them to understand the motor's behavior, characteristics and limits.

More precisely, we consider a motor with a gearbox of ratio $\rho$, and redefine $\tau$ and $\omega$ as the output (articular) parameters, such that \eqref{eFullModel} becomes

\begin{equation}
	\begin{aligned}
			&\left\{
			\begin{aligned}
				L \frac{d i_d}{dt} &= - R i_d + n_p \rho \omega  L i_q + u_d \\
				L \frac{d i_q}{dt} &= - R i_q - \rho \omega (n_p L i_d + k_e) + u_q\\
				\tau &= \frac{3}{2} \rho k_e i_q
			\end{aligned}
			\right.\\
			\text{with:}
			&\left\{
			\begin{aligned}
				i_q^2 + i_d^2 &\leq 2 {I_m}^2 \\
				u_q^2 + u_d^2 &\leq \frac{U_{bat}^2}{3}
			\end{aligned}
			\right.
	\end{aligned}
	\label{eFullModelArticular}
\end{equation}


\subsection{Directly derived constants}

Here, we compute a bunch of simple constants, which are often present in motor datasheet, and give some information about the motor's behavior:

\begin{itemize}
	\item \textbf{Back EMF constant, phase to phase}: this constant is useful as it directly gives the no-load maximum speed:
	\begin{equation}
	k^{pp}_e \triangleq \sqrt{3} k_e \qquad \frac{V}{rad/s}
	\end{equation}
	\item \textbf{Maximum (no load) speed} (articular): more details is given in Section~\ref{sMaxSpeed}
	\begin{equation}
	\omega_{nl} \triangleq \frac{U_{bat}}{\rho k^{pp}_e} = \frac{U_{bat}}{\sqrt{3} \rho k_e}  \quad rad/s
	\end{equation}
	\item \textbf{Torque constant} (articular): proportionality ratio between quadrature current and torque:
	\begin{equation}
	k^q_t \triangleq \frac{3}{2} \rho k_e \qquad \frac{Nm}{A}
	\eqref{eKt}
	\end{equation}

	\item \textbf{Motor constant}(articular): this indicates how much heat is dissipated for a given torque: indeed, the thermal power is $P_th = R (i_a^2 + i_b^2 + i_c^2) \triangleq \frac{1}{K_m^2} \tau$ - this yields
	\begin{equation}
		K_m \triangleq \sqrt{\frac{2}{3}} \frac{k^q_t}{\sqrt{R}} = \sqrt{\frac{3}{2}}  \frac{\rho k_e}{\sqrt{R}} \qquad \frac{Nm}{\sqrt{W}}
	\end{equation}
	\item \textbf{Maximum torque} (articular): 
	\begin{equation}
		\tau_m \triangleq \frac{k^t_q}{\sqrt{2}} I_m = \frac{3}{2 \sqrt{2}} \rho k_e I_m \qquad Nm
	\end{equation}

	\item \textbf{Defluxing ratio}: in French \emph{réaction d'induit}, this ratio is an approximation of how much the speed can be increased by defluxing.
	\begin{equation}
	r_{dflux} \triangleq \frac{n_p L I_m}{\sqrt{2} k_e}
	\end{equation}
\end{itemize}

\subsection{Maximum speed and defluxing} \label{sMaxSpeed}


The motor speed is limited by the input voltage: indeed, induction forces and resistive loss limit the voltage at high velocity and high torque.

To obtain the limit velocity, the idea is to neglect the $L \frac{d}{dt}$ term in \eqref{eFullModelArticular}, to obtain the point where, at constant torque, the motor may operate. $L$ is indeed very small ; more importantly, we are here talking about the quadrature current, not the phase current (which oscillates rapidly).

Under this condition, $u_d$ and $u_q$ are algebraically given as a function of the current:
\begin{equation}
\begin{aligned}
	u_d &= R i_d - n_p \rho \omega  L i_q \\
	u_q &= R i_q + \rho \omega  (n_p L i_d + k_e)\\
\end{aligned}
\label{eVelocityVoltage}
\end{equation}

\subsubsection{Maximum speed when not defluxing}

When no defluxing is in effect, $i_d = 0$. Then the voltage inequality in \eqref{eFullModelArticular} translate into:
\begin{equation}
	(n_p \rho \omega  L i_q) ^2 + (R i_q + \rho \omega  k_e)^2 \leq \frac{U_{bat}^2}{3}
\end{equation}

We thus obtain a second-order polynomial: $a \omega^2 + b \omega + c = 0$, with

\begin{equation}
	a = \rho^2 ((n_p L i_q)^2 + k_e^2) \qquad 
	b = 2 \rho R k_e i_q \qquad 
	c = (R i_q)^2 - \frac{U_{bat}^2}{3}
\end{equation}

This polynomial typically has two real roots. One of them is negative, and has little practical interest: it corresponds to the moment where, in generator mode, the phase voltage is equal to the battery voltage, while the motor is braking at maximum torque. This will never happen - unless an excessively large load is able to so thoroughly over-power the motor. Note that for high currents, the roots might be imaginary: again this is non-physical. Indeed, before the discriminant becoming negative, at one point the maximum velocity will become negative. This indicates that the motor is no longer able to turn - physically, it's probably because the voltage drop in the phase resistor becomes as large as the battery voltage itself.

Thus, the maximum velocity at a given torque is given by the positive root of this equation. Note that for $i_q=0$, this simplifies into: $ \omega = \frac{U_{bat}}{\sqrt{3}\rho k_e} = \frac{U_{bat}}{\rho k^{pp}_e}$.

\subsubsection{Defluxing}

As we have seen in the previous section, the velocity limit arises due to a voltage limit, cause by the induction term $\omega k_e$. The idea of defluxing is to send a negative direct current $i_d$: this current will not modify the torque, but will reduce the induction term $\omega (n_p L i_q + k_e)$. This will effectively reduce the quadrature voltage $u_q$, but, due to the resistor $R$, will also increase the direct voltage $u_d$: this limits the defluxing capability of a given motor.

Another limit is the current limit, which implies that it is not always possible to completely cancel out the flux rotor $k_e$. The so-called defluxing ratio (\emph{réaction d'induit}) $r_{dflux} \triangleq \frac{n_p L I_m}{\sqrt{2} k_e}$ indicates this capacity. A motor will a defluxing ratio smaller than one cannot completely cancel out the magnetic flux, even at zero torque, and thus has a finite maximum velocity. By contrast, a motor with a defluxing ratio larger than or equal to one can do it - and in theory has an infinite velocity at zero torque (thus a finite mechanical power nonetheless). Note of course that in practice, mechanical friction will limit the effective velocity.

\bigskip 

\paragraph{Defluxing current}

Given a target quadrature current $i_q$ (i.e. target torque) and a target velocity $\omega$, we compute the required defluxing current as the current that satisfies the voltage inequality:
\begin{equation}
	(R i_d - n_p \rho \omega L i_q) ^2 + (R i_q + \rho \omega (n_p L i_d + k_e))^2 = \frac{U_{bat}^2}{3}
	\label{eIneqDeflux}
\end{equation}

Once again, we get a second-order polynomial in $i_d$, with coefficients

\begin{equation}
	\begin{aligned}
	a &= R^2 + (\rho \omega n_p L)^2 \qquad 
	b = 2 n_p L k_e (\rho \omega)^2 \\ 
	c &= (\rho \omega n_p L i_q)^2 + 2 R i_q k_e \rho \omega + R^2 i_q^2 + (k_e \rho \omega)^2 - \frac{U_{bat}^2}{3}
	\end{aligned}
\end{equation}

We typically obtain two real roots. If the largest is positive, this indicates that there is no need to perform defluxing: indeed, to reach the maximum voltage we need to add \emph{positive} direct current. This is of course not the point (at is would only contribute to increasing the power dissipation), thus in that case we do not deflux. If both roots are negative, this gives a range of applicable defluxing current - there again, we take the largest root (smallest in absolute value) to avoid unnecessary power loss. It is however worth mentioning that this presence of multiple solution means that the deflux current is not unique: rather, it is only a minimum. In typical cases this only means that we need to put \emph{at least} this much current for the motor to spin at the desired speed - but we can put more. This is a nice property, given that we have some uncertainty on the actual parameters of a given motor: this shouldn't impact our capacity to deflux it.

To summarize, the defluxing current needed to reach a given torque and speed writes:

\begin{equation}
	i_d = \min\left(0, \frac{-b + \sqrt{b^2 - 4 a c}}{2 a}\right)
\end{equation} 


\paragraph{Maximum speed}

In practice, the defluxing current cannot be as large as we want, as we must still satisfy the current inequality in \eqref{eFullModelArticular}. Thus, considering a given torque, thus a given $i_q$, a velocity limit is still present. 

We can obtain a closed-form solution of this maximum velocity by noticing that, in \eqref{eIneqDeflux}, the right hand side is minimal when $i_d = -\frac{k_e}{n_p L}$\footnote{This is clear if $i_d < 0$ ; considering that on a real motor, $R < \rho \omega n_p L$, the voltage also increases when applying a positive direct current.}. Thus, we can compute the maximum defluxing current as

\begin{equation}
	i_d = \max \left(-\frac{k_e}{n_p L}, -\sqrt{2 I_m^2 - i_q^2} \right)
\end{equation}

Then, given $i_d$ and $i_q$, \eqref{eIneqDeflux} is again a second-order polynomial in $\omega$, whose positive root gives the maximum velocity. Its coefficients are:

\begin{equation}
	\begin{aligned}
		a &= (\rho n_p L i_q) + \rho^2(n_p L i_d + k_e)^2 \qquad 
		b = 2 \rho R i_q k_e \\ 
		c &= R^2 (i_d^2 + i_q^2) - \frac{U_{bat}^2}{3}
	\end{aligned}e
\end{equation}


\end{document}